
\subsubsection{【概要】}
hikiで書いたのを中間発表の予稿集用に変換するマニュアル.

\subsubsection{【下準備】}\begin{enumerate}
\item yahataProjectをinstall
\item rake initを実行.passwordを聞かれたときはdonkeyかnishitaniに知らせる
\item setenvでpathを設定
\end{enumerate}
\subsubsection{【手順】}\begin{enumerate}
\item hikiにGraduate\_handout原稿を作成
\item 作業用のdirectoryを作成
\item そこで,hiki2abst.rb ~/Sites/hiki-1.0/data/text/Graduate\_handout > Graduate\_handout.tex
\item open Graduate\_handout.tex
\item command-Tでpdfへ変換.
\end{enumerate}
\subsubsection{【graph作成手順】}
\paragraph{hiki}\begin{enumerate}
\item 適当なサイズに調整したファイルを添付ファイルとしてUpLoad
\end{enumerate}\begin{itemize}
\item ImageMagickがinstallされているとして,convert A.png -scale 50% B.png
\end{itemize}
\paragraph{latex directory}\begin{enumerate}
\item hikiでファイルを選択してDownLoad
\item 作業用のlatex directoryへ移す.
\item ebb B.pngでB.bbを作成.
\item TeXShopでcompileすればいけるはず.
\end{enumerate}
\subsubsection{【TeXShopの設定】}
\paragraph{ファイルが文字化け}\begin{itemize}
\item 「設定」->書類->エンコーディング Unicode(UTF-8)
\item  同じタグの「設定プロファイル」-> pTeX(ptex2pdf)を選択
\item TeXShopを再起動
\item compileすればいけるはず.
\end{itemize}
\paragraph{compileうまくいかないとき.}\begin{itemize}
\item *.auxファイルを消去して,再compile.
\end{itemize}
