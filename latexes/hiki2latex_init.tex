
\subsubsection{【変更作業の手順】}\begin{itemize}
\item hikiをhtmlに変換するhikidoc.rbに加えていく形で変更
\item hikidocについて調べよ(澄田の宿題)
\end{itemize}\begin{lstlisting}[style=customRuby]
bob% diff hikidoc.rb master-hikidoc/lib/hikidoc.rb 
39d38
< require './hiki2latex.rb'
58,61d56
<   def HikiDoc.to_latex(src, options = {})
<     new(LatexOutput.new(""), options).compile(src)
<   end
< 
916,917c911
< #  puts HikiDoc.to_html(ARGF.read(nil))
<   puts HikiDoc.to_latex(ARGF.read(nil))
---
>   puts HikiDoc.to_html(ARGF.read(nil))
\end{lstlisting}\begin{itemize}
\item hiki2latex.rbに変更を加えていく.
\end{itemize}
\subsubsection{【設計】}
\paragraph{【maketitleの挿入】}
latex formatに仕込む際にテキストの順序が前後するコマンドが幾つかある.
\\maketitleの前に置かれた,
\begin{itemize}
\item title
\item author
\end{itemize}
である.これらはheadersとしてtext部と別に返すことも可能である.
具体的には,

\begin{table}[htbp]\begin{center}
\caption{}
\begin{tabular}{lll}
\hline
to\_html   &*fを返す  \\ \hline
to\_latex   &*fと*headを返す.  \\
\hline
\end{tabular}
\label{default}
\end{center}\end{table}
%for inserting separate lines, use \hline, \cline{2-3} etc.

なおto\_htmlとの互換性を維持するため*headを後に返すようにしている.

しかし,この部分は,to\_htmlとの整合性を考えるとto\_latexで処理してしまった方がよい.
そこで,
\begin{lstlisting}[style=customRuby]
def finish
  @f.string
end
\end{lstlisting}
となっているのを,
\begin{lstlisting}[style=customRuby]
  def finish
    if @head != "" then
      @head << "\\maketitle\n"
      return @head+@f
    else
      return @f
    end
  end
\end{lstlisting}
とした.

\subsubsection{【sample】}\begin{itemize}
\item \verb|Jihou15|
\end{itemize}
\subsubsection{【pdfの貼り込み】}
pdfpagesを使おうとすると
\begin{quote}\begin{verbatim}
/usr/local/texlive/2015/texmf-dist/tex/latex/pdfpages/pdfpages.sty:70: LaTeX Er
ror: Missing \begin{document}.

See the LaTeX manual or LaTeX Companion for explanation.
Type  H <return>  for immediate help.
 ...                                              
                                                  
l.70 \input{pp\AM@driver.def}
\end{verbatim}\end{quote}
というエラーが出る.この解決法がわからず,incudegraphicsで対応.
\begin{quote}\begin{verbatim}
¥documentclass{jsarticle}
¥usepackage[dvipdfmx,hiresbb]{graphicx}
¥begin{document}
¥includegraphics[width=5cm]{sample.pdf}
¥end{document}

の4行目を
¥includegraphics[page=3,width=5cm]{sample.pdf}
とすれば、

普通に3ページ目の画像を使うことができました。

美文書5版のCD-ROMから普通にインストールした TeXShopの環境です。

つまり、何も(といっても、同書 P114 にあるように、
 --shell--escape--commands=extractbb 
をTeXShop環境設定の「内部設定」タブ pdfTex のLatexのオプションとして付け加えています。)特別なことをせずに、ページ指定をするだけで、できてしまいました。

次の、美文書の版には、ページ設定についても解説していただくといいかも知れません。>奥村先生お願いします。
\end{verbatim}\end{quote}
